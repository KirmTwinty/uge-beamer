\documentclass[compress, aspectratio=169]{beamer}

\usetheme{uge}

\title{Title of the presentation, that can spread over multiple lines, made for long titles.}
\date{6 novembre 2019}
\conference{$27^{\textrm{ème}}$ congrès Société Française de Thermique}

\author[author1@univ-some.com]{Author 1}{1}
\author[author2@univ-some.com]{Author 2}{1}
\author[author3@univ-else.com]{Author 3}{1,2}
\author[author4@univ-else.com]{Author 4}{2}
\author[]{Author 5}{2}
\affil[1]{University of Somewhere, in some country}
\affil[2]{University of Elsewhere, in another country}


%\author[opt1]{Author 1}[opt2]
%\author[test]{Author 2}[ljdf]
%\author[2]{Author 3}[dmf]
%\subtitle{Subtitle of the presentation}
%% Redefine the logo
%\titlelogo[width=3cm,keepaspectratio]{img/logo_white.eps}
\begin{document}


\begin{frame}
  \titlepage
\end{frame}

\section{First section}
\subsection{1}
\begin{frame} 
\frametitle{There Is No Largest Prime Number} 
\framesubtitle{The proof uses \textit{reductio ad absurdum}.} 
\begin{theorem}
There is no largest prime number. \end{theorem} 
\begin{enumerate} 
\item<1-| alert@1> Suppose $p$ were the largest prime number. 
\item<2-> Let $q$ be the product of the first $p$ numbers. 
\item<3-> Then $q+1$ is not divisible by any of them. 
\item<1-> But $q + 1$ is greater than $1$, thus divisible by some prime
number not in the first $p$ numbers.
\end{enumerate}
\end{frame}

\subsection{2}
\begin{frame}{A longer title}
\begin{itemize}
\item one
\begin{itemize}
  \item two
  \begin{itemize}
  \item three
  \end{itemize}
  \end{itemize}
\end{itemize}
\end{frame}

\section{Second section}
\subsection{3}
\begin{frame}{Another slide}
Some blabla
\end{frame}


\end{document}

%%% Local Variables: 
%%% mode: latex
%%% TeX-engine: xetex
%%% End: