\documentclass[compress, aspectratio=169]{beamer}
\synctex=1
% Load the theme see the README for optional parameters
\usetheme{uge}

% Define the title of the presentation
\title{Title of the presentation, that can spread over multiple lines, made for long titles.}

% You can also give a subtitle, or not
\subtitle{You can even add a subtitle}

% Put the date
\date{September, 3rd 2020}

% The conference name
\conference{$42^{\textrm{nd}}$ Conference}

% List of authors such as:
%  \author[mail adress (optional)]{Author name}{Affiliation index}
\author[author1@univ-some.com]{Author 1}{1}
\author[author2@univ-some.com]{Author 2}{1}
\author[author3@univ-else.com]{Author 3}{1,2}
\author[author4@univ-else.com]{Author 4}{2}
\author{Author 5}{2}

% List of affiliations such as:
%  \affil[affiliation number]{Name of affiliation}
\affil[1]{University of Somewhere, in some country}
\affil[2]{University of Elsewhere, in another country}


\begin{document}

%% Generate Titlepage
\begin{frame}
  \titlepage
\end{frame}

\begin{frame}
  \tableofcontents
\end{frame}


\section{First section}
\subsection{Theorem example}
\begin{frame} 
\frametitle{There Is No Largest Prime Number} 
\framesubtitle{The proof uses \textit{reductio ad absurdum}.} 
\begin{theorem}
There is no largest prime number. \end{theorem} 
\begin{enumerate} 
\item<1-| alert@1> Suppose $p$ were the largest prime number. 
\item<2-> Let $q$ be the product of the first $p$ numbers. 
\item<3-> Then $q+1$ is not divisible by any of them. 
\item<4-> But every number $> 1$ has a prime divisor
\item<1-> Thus there is a prime number greater than p
\end{enumerate}
\end{frame}

\subsection{Some long title}
\begin{frame}{A very long title to show that you have some space to write down your ideas}
\begin{itemize}
\item one
\begin{itemize}
  \item two
  \begin{itemize}
  \item three
  \end{itemize}
  \end{itemize}
\end{itemize}
\end{frame}

\section{Second section}
\subsection{Equation example}
\begin{frame}{Another slide}
  Some equation:
  \begin{equation*}
    \frac{\partial\phi}{\partial t} = \nabla \cdot \left(\phi V\right) = S
    \label{eq:1}
  \end{equation*}

  
\end{frame}



\end{document}

%%% Local Variables: 
%%% mode: latex
%%% TeX-engine: xetex
%%% End: